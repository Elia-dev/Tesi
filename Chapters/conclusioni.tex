\chapter{Conclusioni}
Giunti alla fase conclusiva della mia tesi, in cui è stato trattato in modo completo l'argomento seL4, ed essendo ormai chiaro cosa sia un sistema operativo, credo sia opportuno proporre alcune osservazioni.

Partendo dal presupposto che, verosimilmente le persone a conoscenza dell'esistenza di seL4 siano molto poche, dato che è evidente che non sia tanto noto quanto i sistemi operativi Windows, Mac o Linux e, come dimostrato nei capitoli precedenti del mio elaborato, si tratti di un sistema efficiente e sicuro, penso che sia stato inevitabile, durante la lettura, domandarsi il motivo per cui si sia sentito nominare questo \textit{microkernel} per la prima volta.

Analizzando diversi elementi, si può affermare che, essendo seL4 un \textit{microkernel}, i servizi che può offrire sono davvero pochi; pensare di realizzare un sistema operativo che ricalchi i tre sopra menzionati, altamente \textit{user-friendly}, diffusi su scala globale ed utilizzabili nella maggior parte delle situazioni, risulterebbe davvero complesso, considerato l'elevato numero di elementi da sviluppare partendo da 0. Altro aspetto certamente non secondario è la sicurezza: per la distribuzione di massa non è necessario avere un livello di protezione così elevato; quando si tratta di salvaguardia dei dati, non esiste un vero e proprio limite, specialmente oggi, in un contesto globale in cui "i dati sono il nuovo petrolio"\footnote{Clive Humby, 2006}, ma è assolutamente necessaria una valutazione costi-benefici. Riporto il caso pratico della gestione degli accessi di ogni singolo file: per un utente finale meno esperto ciò renderebbe il sistema inutilizzabile. Prendendo come esempio Linux, poche persone al di fuori dell'ambito scientifico/informatico sono in grado di utilizzarlo; se si vanno poi ad aggiungere regole più stringenti nella gestione dei permessi, la conclusione è dunque lampante: sarà molto difficile realizzare un prodotto di utilizzo comune.

Con questa mia osservazione non voglio escludere la possibilità che un giorno possa accadere, ma chiaramente richiederebbe un enorme impegno, ed essendo seL4 \textit{open-source}, questo implicherebbe la formazione e la crescita di una \textit{community} disposta a svilupparlo e a mantenerlo aggiornato, come accade oggi per Linux.

Dunque vediamo qualche utilizzo che è stato fatto di seL4. Ci sono stati vari progetti e ce ne sono altri ancora in corso d'opera, per semplicità ne nominerò alcuni che secondo me sono i più rilevanti. 

Un'importante collaborazione è stata quella con il DARPA (\textit{Defense Advanced Research Projects Agency}), in particolare per il progetto HACMS (\textit{High-Assurance Cyber Military Systems}). Consiste nello sviluppo di mezzi sia terrestri sia aerei altamente sicuri tanto da poterli considerare "inattaccabili".

Un primo esempio che voglio portare è l'elicottero a guida autonoma \textit{Boeing Little Bird helicopter}, il quale ha sorvolato una certa area senza pilota; simultaneamente è stato fatto un \textit{test} sulla sicurezza, per cui durante il volo è stato eseguito un attacco \textit{hacker}, che in parte ha avuto successo. Gli \textit{hacker} sono riusciti ad avere accesso al computer di bordo ma è stato impossibile disabilitarlo, anche tentanto di mandare in \textit{crash} il sistema.

Per quanto riguarda i mezzi terrestri, è stato realizzato un robot da esplorazione \textit{TARDEC GVR-Bot} e un camion militare \textit{TARDEC AMAS}. Entrambi sono mezzi a guida autonoma, anche in questo caso, con del \textit{software} basato su seL4, il quale è stato combinato o sostituito in alcuni parti del sistema, per aumentarne la sicurezza.

Un ulteriore progetto, molto giovane, a parere mio rilevante, è quello del nuovo sistema operativo \textit{KataOS}. Tale SO è sviluppato da \textit{Google} ed ha diverse componenti \textit{open-source}; l'obiettivo è quello di creare un sistema operativo sicuro nell'ambito dell'\textit{internet} delle cose e del \textit{machine learning}. La garanzia di sicurezza è dovuta principalmente dall'utilizzo di seL4 come \textit{microkernel} su cui si basa l'intero sistema.

Queste sono solo alcune, tra le più note, implementazioni di seL4 di cui si hanno notizie documentate. Considerata la sicurezza e l'affidabilità mi auguro in un futuro di vedere l'utilizzo del \textit{microkernel} anche al di fuori del campo militare. L'ambito medico, spaziale e la ricerca scientifica - dove chiaramente questi due elementi sono fondamentali - essi si trovano una fase di evoluzione importante in questo periodo storico, l'introduzione di seL4 permetterebbe alle nuove tecnologie una base solida su cui basarsi; la medesima considerazione la si può applicare alla recente introduzione della guida autonoma, sia nei mezzi di trasporto pubblici sia nei privati dove sicurezza e affidabilità sono imperativi.
\newpage

Questa tesi, oltre a sperimentare sul \textit{microkernel}, ha anche lo scopo di proporsi come mezzo di diffusione di seL4. Come ho detto poco sopra dobbiamo essere realisti, optare per l'utilizzo del \textit{microkernel} non è una cosa facile, c'è molto lavoro da fare, ma se questo può portare alla creazione di sistemi veloci, sicuri ed affidabili probabilmente ne vale la pena investire tempo ed energie. E perché no, magari anche dei sistemi operativi di uso comune, come quello che permette l'utilizzo del vostro \textit{smartphone} che custodite in tasca oppure del vostro \textit{computer}.
