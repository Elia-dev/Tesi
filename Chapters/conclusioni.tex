\chapter{Conclusioni}
Giunti alla fase conclusiva della mia tesi, in cui è stato trattato in modo completo l'argomento seL4 ed essendo ormai chiaro cosa sia un sistema operativo, credo sia opportuno proporre alcune osservazioni.

Partendo dal presupposto che, verosimilmente le persone a conoscenza dell'esistenza di seL4 siano molto poche, dato che è evidente che non sia tanto noto quanto i sistemi operativi Windows, Mac o Linux e, come dimostrato nei capitoli precedenti del mio elaborato, si tratti di un sistema efficiente e sicuro, penso che sia stato inevitabile, durante la lettura, domandarsi il motivo per cui si sia sentito nominare questo \textit{microkernel} per la prima volta.

Analizzando diversi elementi, si può affermare che, essendo seL4 un \textit{microkernel}, i servizi che può offrire sono davvero pochi; pensare di realizzare un sistema operativo che ricalchi i tre sopra menzionati, altamente \textit{user-friendly}, diffusi su scala globale ed utilizzabili nella maggior parte delle situazioni, risulterebbe davvero complesso, considerato l'elevato numero di elementi da sviluppare partendo da 0. Altro aspetto certamente non secondario è la sicurezza: per la distribuzione di massa non è necessario avere un livello di protezione così elevato; quando si tratta di salvaguardia dei dati, non esiste un vero e proprio limite, specialmente oggi, in un contesto globale in cui "i dati sono il nuovo petrolio"\footnote{Clive Humby, 2006}, ma è assolutamente necessaria una valutazione costi-benefici. Riporto il caso pratico della gestione degli accessi di ogni singolo file: per un utente finale meno esperto ciò renderebbe il sistema inutilizzabile. Prendendo come esempio Linux, poche persone al di fuori dell'ambito scientifico/informatico sono in grado di utilizzarlo; se si vanno poi ad aggiungere regole più stringenti nella gestione dei permessi, la conclusione è dunque lampante: sarà molto difficile realizzare un prodotto di utilizzo comune.

Con questa mia osservazione non voglio escludere la possibilità che un giorno possa accadere, ma chiaramente richiederebbe un enorme impegno, ed essendo seL4 \textit{open-source}, questo implicherebbe la formazione e la crescita di una \textit{community} disposta a svilupparlo e a mantenerlo aggiornato, come accade oggi per Linux.

Dunque vediamo qualche utilizzo che è stato fatto di seL4. Ci sono stati vari progetti e ce ne sono altri ancora in corso d'opera, per semplicità ne nominerò alcuni che secondo me sono i più rilevanti. Un'importante collaborazione è stata quella in collaborazione con il DARPA (\textit{Defense Advanced Research Projects Agency}), in particolare per il cosiddetto \textit{High-Assurance Cyber Military Systems project}). Questo progetto consiste nello sviluppo di mezzi sia terrestri sia aerei altamente sicuri tanto da voler realizzare sistemi "inattaccabili"


%https://www.ilsoftware.it/kataos-sistema-operativo-sicuro-per-dispositivi-%iot-e-machine-learning_25132/

%https://www.newscientist.com/article/mg22730392-600-unhackable-kernel-could-%keep-all-computers-safe-from-cyberattack-2/

