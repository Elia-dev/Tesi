\chapter{Conclusioni}
Giunti alla fase conclusiva della mia tesi, avendo in mano un quadro completo di seL4 e di cosa sia un sistema operativo, ritengo sia opportuno proporre delle osservazioni.

Partendo dal presupposto che verosimilmente le persone a conoscenza dell'esistenza di seL4 sono molto poche, ne deriva che non è conosciuto come i più noti sistemi operativi Windows, Mac o Linux; d'altra parte nello svolgimento del mio elaborato è chiaramente emerso che si tratta di un sistema efficiente e sicuro, e che quindi sia lecito che durante la lettura sia sorta la questione del perché sia la prima volta che si sente nominare questo \textit{microkernel}.

Valutiamo tutti gli elementi ormai a nostra disposizione: seL4 è un \textit{microkernel}, quindi i servizi che vengono offerti da esso sono estremamente pochi; pensare di realizzare un sistema operativo tipo i 3 nominati poco sopra, altamente \textit{user-friendly}, diffusi su scala globale e utilizzabili nella maggior parte dei contesti, risulta veramente difficile, considerando quante cose sarebbe necessario sviluppare da zero. Un altro elemento da considerare è la sicurezza: non è necessario avere un livello di sicurezza così elevato per la distribuzione di massa. Quando si parla di sicurezza dei dati non esiste un vero e proprio limite, specialmente oggi in cui "i dati sono il nuovo petrolio"\footnote{Clive Humby, 2006}, ma comunque è necessario fare una valutazione costi-benefici. Dover gestire, ad esempio, gli accessi di ogni singolo \textit{file}, per l'utente finale meno esperto questo renderebbe il sistema inutilizzabile. Prendiamo ora come esempio Linux, poche persone al di fuori dell'ambito scientifico/informatico sono capaci di utilizzarlo, aggiungiamo regole più stringenti per quanto riguarda la gestione dei permessi e presto segue la conclusione che sarà molto difficile realizzare un prodotto di utilizzo comune.

