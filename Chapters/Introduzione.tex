\chapter{Introduzione}
In questa tesi andremo ad affrontare uno studio del \textit{microkernel seL4}, sia da un punto di vista descrittivo sia trattando gli aspetti più tecnici.

Un primo passo che ho fatto verso la scrittura di questo testo è stato quello di fare una ricerca sulla letteratura che si trova disponibile riguardo a seL4, nonostante sia poca e principalmente fornita dalla \textit{seL4 Foundation} stessa è comunque stata sufficiente per avere una conoscenza abbastanza approfondita del \textit{microkernel}.

Dunque ho iniziato a farmi un'idea generale dell'argomento attraverso il whitepaper \cite{sel4-whitepaper} di presentazione il quale a grandi linee dà una visione a trecentosessanta gradi di seL4 entrando in alcune parti anche abbastanza nello specifico e nel tecnico. Ovviamente accompagnato a questa lettura ho dovuto fare un approfondimento sul \textit{kernel} e comunque in generale sui sistemi operativi, in quanto prima di approcciarmi a questa tesi non avevo fatto uno studio così specifico sul \textit{kernel} e in particolare sui \textit{microkernel}.

Nel mio percorso universitario avevo già affrontato un corso che trattava i sistemi operativi ma questa esperienza mi ha fatto vedere la materia da un'angolatura diversa. Quasi tutte le conoscenze che sono state necessarie per fare questo studio erano già acquisite ma spesso queste erano fini a se stesse. Studiare nelle specifico un singolo sistema mi ha permesso di capire come tutti i vari pezzi cooperano e si incastrano come in un puzzle per arrivare al disegno finale.

Dopo queste prima fase di studio mi sono cimentato più sugli aspetti tecnici, quindi guardando ed esaminando le chiamate si sistema, le varie funzioni che sono disponibili nelle API (\textit{Application Programming Interface)}, anche mettendo mano direttamente su di esse. In particolare sono andato a svolgere dei tutorial che la \textit{seL4 Foundation} mette a disposizione appositamente per permettere a chi si approccia al \textit{microkernel} di prendere confidenza con esso. Questi tutorial sono divisi in sezioni e ognuno di essi tratta un argomento diverso; consistono in un programma, in parte non funzionante, che va prima di tutto capito e successivamente sistemato in modo che funzioni correttamente.

Ovviamente questa seconda fase è stata quella più duratura e più formativa, in quanto è molto complicato per chi non ha mai fatto programmazione a livello \textit{kernel} fare questo tipo di esperienza. Oltretutto questi esercizi sono strutturati in modo che venga fatto un approfondimento specifico sulla parte trattata da quell'esercitazione; ad esempio l'utilizzo della memoria virtuale: come fare il \textit{mapping}, le varie fasi intermedie e tutte le funzioni che permettono di utilizzarla. In questo stadio ho quindi avuto modo di mettere in pratica il mio bagaglio di conoscenze sul campo vedendo effettivamente come si implementano certi processi.
\newline

Nel proseguimento della lettura seguirà una parte iniziale (capitolo 2) che serve che per dare delle conoscenze di base sui sistemi operativi così da poter creare un background di conoscenze che dovrebbero essere sufficienti per la comprensione del resto della tesi. 

Successivamente ci sarà un capitolo 3 che entrerà nello specifico del \textit{microkernel} seL4, quindi ne farà una descrizione generale del funzionamento parlandone sia da un punto di vista descrittivo ma anche entrando nel tecnico di alcuni meccanismi.

Nel quarto capitolo invece affronteremo una fase di sperimentazione sul \textit{microkernel}, ci saranno una serie di sorgenti C con vari errori che dovranno essere corretti in modo da renderli funzionanti. Questi programmi utilizzano oggetti e funzioni altamente specifici per seL4.