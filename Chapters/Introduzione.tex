\chapter{Introduzione}
In questa tesi andremo ad affrontare uno studio del \textit{microkernel seL4}, sia da un punto di vista descrittivo sia trattando gli aspetti più tecnici.

Un primo approccio che ho avuto è stato quello di fare una ricerca sulla letteratura che si trova disponibile riguardo a seL4, nonostante sia poca e principalmente fornita dalla \textit{seL4 Foundation} stessa è comunque stata sufficiente per avere una conoscenza abbastanza approfondita del \textit{microkernel}. 


seL4 fa parte della famiglia dei \textit{microkernel} L4 che risalgono alla prima metà degli anni '90, creato da Jochen Liedtke per sopperire alle scarse performance dei primi sistemi operativi basati su \textit{microkernel}. seL4 in particolare è stato sviluppato dal gruppo NICTA oggi conosciuto con il nome di Trustworthy System.

