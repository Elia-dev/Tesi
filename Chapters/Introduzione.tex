\chapter{Introduzione}
In questa tesi andremo ad affrontare uno studio del \textit{microkernel seL4}, sia da un punto di vista descrittivo sia da un punto di vista più tecnico.

Il punto di partenza per la scrittura di questo elaborato è stato quello di svolgere una ricerca sulla letteratura disponibile riguardo a seL4, nonostante sia poca e principalmente fornita dalla \textit{seL4 Foundation} stessa, \textcolor{green}{risulta} comunque sufficiente per avere una conoscenza abbastanza approfondita del \textit{microkernel}.

Ho quindi iniziato a farmi un'idea generale dell'argomento attraverso il \textit{whitepaper} \cite{sel4-whitepaper} di presentazione, il quale a grandi linee dà una visione a trecentosessanta gradi di seL4, entrando in alcune parti anche in ambito specifico e tecnico. Ovviamente in parallelo a questa lettura ho dovuto affiancare un approfondimento sul \textit{kernel} e in generale sui sistemi operativi, in quanto prima di approcciarmi a questa tesi non avevo una conoscenza così specifica sul \textit{kernel} e in particolare sui \textit{microkernel}.

Nel mio percorso universitario avevo già affrontato un corso che trattava i sistemi operativi, ma questa esperienza mi ha dato la possibilità di vedere la materia da un'angolatura diversa. Quasi tutte le conoscenze che sono state necessarie per intraprendere questa ricerca erano già state acquisite ma spesso, quando si studia a livello teorico, capita che alcuni concetti siano magari chiari, ma fini a se stessi. Indagare nello specifico un singolo sistema mi ha permesso di capire come tutti i vari pezzi cooperano e si incastrino, come in un puzzle per arrivare al disegno finale.

Dopo queste prima fase di analisi mi sono cimentato più sugli aspetti tecnici, guardando ed esaminando quindi le chiamate di sistema, le varie funzioni che sono disponibili nelle API (\textit{Application Programming Interface)}, mettendo anche mano direttamente su di esse. In particolare sono andato a svolgere dei \textit{tutorial} che la \textit{seL4 Foundation} mette a disposizione appositamente per permettere a chi si approccia al \textit{microkernel}, di prendere confidenza con esso. Questi \textit{tutorial} sono divisi in sezioni e ognuno di essi tratta un argomento diverso; consistono in un programma, in parte non funzionante, che va prima di tutto compreso e successivamente modificato, in modo che funzioni correttamente.

Ovviamente questa seconda fase è stata quella più duratura e più formativa, in quanto è molto complicato, per chi non ha mai programmato a livello \textit{kernel}, fare questo tipo di esperienza. Questi esercizi sono oltretutto strutturati in modo tale da consentire un approfondimento specifico sulla parte trattata da quell'esercitazione; ad esempio l'utilizzo della memoria virtuale: come fare il \textit{mapping}, le varie fasi intermedie e tutte le funzioni che permettono di utilizzarla. In questa fase ho quindi avuto modo di mettere in pratica il mio bagaglio di conoscenze, vedendo effettivamente come si implementano certi processi.
\newline

Nel proseguimento della lettura, seguirà una parte iniziale (capitolo 1) che serve per fornire delle conoscenze di base sui sistemi operativi, così da poter creare un \textit{background} di informazioni che dovrebbero essere sufficienti per la comprensione del resto della tesi. 

Successivamente, col capitolo 2 si entrerà nello specifico del \textit{microkernel} seL4, in cui si troverà una descrizione generale del funzionamento, approfondito tecnicamente in tutte le varie parti di cui è composto.

Nell'ultimo capitolo invece si affronterà una fase di sperimentazione sul \textit{microkernel}; ci saranno una serie di sorgenti C con vari errori, che dovranno essere corretti in modo da renderli funzionanti. Questi programmi utilizzano oggetti e funzioni creati appositamente per la gestione dei processi e di tutte le funzionalità di seL4. 
