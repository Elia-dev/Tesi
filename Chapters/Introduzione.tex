\chapter{Introduzione}
A chiusura della prima parte del mio corso di laurea, ho scelto come argomento di tesi uno studio e sperimentazione sul \textit{microkernel} seL4, che nel suo svolgimento, si sviluppa con un taglio sia descrittivo sia tecnico, partendo da una ricerca nella letteratura disponibile riguardo a seL4, nonostante sia scarsa e principalmente fornita dalla seL4 \textit{Foundation} stessa, ma comunque sufficiente per avere una conoscenza abbastanza accurata del \textit{microkernel}.

Per ottenere una visione più completa possibile di seL4, grazie al \textit{whitepaper} \cite{sel4-whitepaper} di presentazione, nella fase iniziale ho quindi fatto un approfondimento in ambito tecnico e specifico su di esso, per poi affiancare - in parallelo a questa lettura - una ricerca mirata sul \textit{kernel} e più in generale sui sistemi operativi. Sebbene avessi affrontato un corso universitario che mirava ad esaminare tali processi, se prima di cominciare questa tesi non possedevo una conoscenza così specifica sul \textit{kernel} e in particolare sui \textit{microkernel}, questa esperienza mi ha offerto la possibilità di vedere la materia in una prospettiva diversa. Infatti, nonostante la maggior parte delle conoscenze essenziali per lo svolgimento del mio studio fossero già state acquisite, credo che spesso, quando si studia a livello teorico capiti che, pur avendo chiari i concetti, essi rimangano fini a se stessi, mentre comprendere a fondo un singolo sistema mi ha concretamente aiutato ad avere evidente come tutti i vari pezzi effettivamente cooperino e si incastrino tra loro, come nel disegno finale di un immaginario puzzle.

La seconda fase del mio piano di lavoro si è poi articolata, concentrandomi maggiormente sugli aspetti tecnici, nell'analisi delle chiamate di sistema, delle varie funzioni disponibili nelle API (\textit{Application Programming Interface}), mettendo anche direttamente mano su di esse e successivamente con lo svolgimento dei \textit{tutorial} che la seL4 \textit{Foundation} mette a disposizione, appositamente per permettere a chi si avvicina al mondo del \textit{microkernel}, di prendere progressivamente confidenza con esso. Tali \textit{tutorial} sono divisi in sezioni e ognuno di questi tratta un argomento diverso; consistono in un programma, in parte non funzionante, che va prima di tutto compreso e successivamente modificato, in un modo tale per cui funzioni correttamente.

Ovviamente questo secondo stadio è stato quello più lungo e impegnativo e al tempo stesso più formativo, dato che è molto complicato, per chi non ha mai programmato a livello \textit{kernel}, sperimentare in questo campo; in aggiunta, questi esercizi sono strutturati in maniera tale da consentire un supporto specifico sulla parte trattata da quella esercitazione; cito l'esempio dell'utilizzo della memoria virtuale: come fare il \textit{mapping}, le varie fasi intermedie e tutte le funzioni che permettono di utilizzarla. In questa fase ho quindi avuto modo di mettere in pratica il mio bagaglio di conoscenze, vedendo effettivamente come si implementano determinati processi.

Nel proseguimento della lettura, seguirà una parte iniziale (capitolo 1) con la funzione di richiamare alcune conoscenze di base sui sistemi operativi, così da poter creare un \textit{background} di informazioni, che dovrebbero essere a mio parere sufficienti per la comprensione del resto della tesi.

Successivamente, col capitolo 2 si entrerà nel dettaglio specifico del \textit{microkernel} seL4, in cui si troverà una descrizione generale del funzionamento, approfondito tecnicamente in tutte le varie parti di cui è composto.

Il terzo capitolo invece sarà dedicato alla descrizione della sperimentazione sul \textit{microkernel}; ci saranno una serie di sorgenti C con vari errori, che dovranno essere corretti in modo da renderli funzionanti. Questi programmi utilizzano oggetti e funzioni creati appositamente per la gestione dei processi e di tutte le funzionalità di seL4.
