\chapter{Impostazione di seL4}
Come primo approccio per arrivare alla scrittura di questa tesi ho innanzitutto fatto una ricerca sulla letteratura che si trova disponibile riguardo a seL4, nonostante sia poca e principalmente fornita da Trustworthy (TS) stesso comunque sufficiente per avere una conoscenza abbastanza approfondita del microkernel.\\
SeL4 è un sistema opensource dunque lo step successivo è stato quello di scaricare seL4 e sperimentare con mano le funzionalità, ovviamente questo ha richiesto un approfondimento più tecnico e specifico, rispetto a quanto fatto finora, di alcuni aspetti come la gestione della memoria fisica e virtuale, l'IPC ecc. che verranno trattati in questo capitolo.

\section{Fase iniziale}
Come prima cosa ho installato sul mio portatile VirtualBox in quanto come consigliato dalle guide fornite da TS sarebbe ottimale lavorare in ambiente Linux, non avendo una partizione del portatile con Linux ho inizialmente pensato di utilizzare una macchina virtuale così da lasciare inalterato il mio computer e comunque avere a disposizione un sistema operativo Linux su cui lavorare. Andando avanti con il set-up del sistema per iniziare a lavorare su seL4 però ho incontrato una prima difficoltà che è stata lo spazio: purtroppo lo spazio nel portatile non era tantissimo, la macchina virtuale, considerando il sistema operativo e l'installazione dei vari prerequisiti per poter far girare il microkernel, cominciava ad occupare molto spazio, dunque ho dovuto cercare un'alternativa; Per sopperire al problema mi sono procurato un SSD su cui sono andato a copiare la partizione creata in VirtualBox continuando la sperimentazione sul microkernel lavorando sull'SSD esterno collegato via USB.
