\chapter{Background}
In questo capitolo sarà presente una prima parte che andrà ad offrire le conoscenze di base minime per comprendere cosa sia un sistema operativo e una breve classificazione di essi. Dopodiché seguirà la descrizione di alcuni concetti, come ad esempio lo \textit{scheduling della CPU}, che sono fondamentali per comprendere il resto dell'elaborato.

\section{Cos'è un sistema operativo}
Un sistema operativo (SO) è un \textit{software} che gestisce le risorse \textit{hardware} e \textit{software} di un sistema di elaborazione, fornendo servizi agli applicativi utente.

In un computer, esso fornisce l'unica interfaccia diretta con l'\textit{hardware} e in quanto tale ha un accesso esclusivo con il massimo dei privilegi detto \textit{kernel mode}. Questo comporta che una vulnerabilità all'interno del sistema operativo può portare a gravi conseguenze per l'integrità e la sicurezza del sistema; inoltre qualche malintenzionato potrebbe approfittare di questo \textit{bug} per trarne illecito profitto.

Uno degli obiettivi principali di un SO è quindi quello di garantire la sicurezza; ulteriore scopo è l'efficienza: un buon sistema operativo deve saper sfruttare al meglio tutte le risorse che ha a disposizione, dalla gestione della memoria per sfruttare in modo ottimale lo spazio, alla schedulazione dei processi per ottimizzare i tempi di esecuzione. Come ultimo obiettivo, ma non per questo meno rilevante, deve rendere il più semplice possibile l'utilizzo del dispositivo su cui è installato.
All'interno di un di SO si può isolare una specifica parte di codice, che è quella che permette al \textit{software} di interfacciarsi con l'\textit{hardware} e quindi l'accesso e la gestione delle risorse di un dispositivo. Questa specifica parte si chiama \textit{kernel}, che come suggerisce il nome (nocciolo dall'inglese), rappresenta la parte centrale di un sistema operativo su cui tutto il resto si appoggia.
\newpage

\section{Architettura software di un sistema operativo}
Esistono vari modelli strutturali per i sistemi operativi: monolitici, modulari, a livelli, \textit{microkernel} ed ibridi. Ad oggi i più diffusi sono gli ibridi, che combinano i vari modelli tra di loro, ma che in gran parte si basano su sistemi monolitici. Quest'ultimi consistono di un unico file binario statico, al cui interno sono definite tutte le funzionalità del \textit{kernel} e che viene eseguito in un unico spazio di indirizzi. Tutto ciò comporta dei vantaggi: 
\begin{itemize}
	\item[-] efficienza: motivo principale per cui la maggior parte dei sistemi operativi ancora oggi si basano su \textit{kernel} in gran parte monolitici. Lavorando nello stesso spazio di indirizzi e gestendo tutto attraverso chiamate a procedura, il SO risulterà molto reattivo e performante;
	\item[-] semplicità: in quanto non ha una vera e  propria strutturazione, bensì il codice è tutto in un unico file binario, risulta chiaramente più semplice da progettare, anche se poi l'implementazione risulta difficile.
\end{itemize} 
D'altra parte ha anche degli svantaggi: 
\begin{itemize}
	\item[-] inserimento di un nuovo servizio: questo richiede la ricompilazione del \textit{kernel}, quindi non permette l'inserimento di un nuovo servizio a \textit{runtime} (problema risolto nei modelli ibridi);
	\item[-] dimensione: dovendo gestire tutte le principali funzionalità del sistema operativo, il \textit{kernel} sarà composto da milioni di righe di codice sorgente (MSLOC - linux ha circa 20MSLOC) e questo porta direttamente al successivo grosso svantaggio;
	\item[-] sicurezza: maggiore è il numero di righe di codice maggiore sarà il numero di possibili \textit{bug}; essendo tutto il codice eseguito nello stesso spazio di indirizzi un \textit{bug} rischia di far bloccare l'intero sistema, anche se il problema è molto piccolo e isolato a una minima funzione del \textit{kernel}.
\end{itemize}

All'estremo opposto troviamo i \textit{microkernel} che sono composti da un \textit{kernel} ridotto al minimo indispensabile, che comprende la gestione della memoria, dei processi e della CPU, le comunicazioni tra processi (IPC) e l'hardware di basso livello; tutto il resto deve essere gestito da server (\textit{daemon}) che operano sopra al \textit{kernel}, quindi in spazi di indirizzi separati.

I \textit{microkernel} sono spesso usati in sistemi \textit{embedded}, in applicazioni \textit{mission critical} di automazione robotica o di medicina, a causa del fatto che i componenti del sistema risiedono in aree di memoria separate, private e protette \cite{kernelWikipedia}.

Anche questo modello ha dei vantaggi:
\begin{itemize}
	\item[-] flessibilità: l'inserimento di un nuovo servizio avviene al di sopra del \textit{kernel} quindi in qualsiasi momento è possibile aggiungere o togliere servizi senza dover modificare il \textit{kernel};
	\item[-] sicurezza: minore quantità di codice eseguita in \textit{kernel mode} (quindi minore quantità di \textit{bug} e minore superficie attaccabile) pertanto maggiore sicurezza del sistema; inoltre i servizi lavorano in uno spazio di indirizzi differente da quello del \textit{kernel}; di conseguenza se un server (su cui viene eseguito un servizio) smette di funzionare, tutto il resto del sistema continua a funzionare normalmente e si potrà procedere a riavviare quel singolo servizio;
	\item[-] semplicità: essendo il codice composto da qualche decina di migliaia di righe di codice (KSLOC) risulta molto più facile da scrivere.
\end{itemize}
Dall'altro lato ha un grande svantaggio:
\begin{itemize}
	\item[-] efficienza: dato che ogni servizio è eseguito a livello utente, l'utilizzo di uno qualsiasi di questi richiede il ricorso a chiamate di sistema, che rallentano fortemente l'esecuzione di ogni operazione, motivo principale per cui ancora oggi i sistemi operativi si basano in gran parte su sistemi monolitici.
\end{itemize}
In Figura~\ref{fig:MonolithicVSmicrokernel} si possono vedere in maniera schematica le differenze tra \textit{kernel} monolitici e \textit{microkernel}.

\begin{figure}[h]
  \includegraphics[width=\linewidth]{img/MonolithicVSmicrokernel.png}
  \caption{\textit{Kernel} monolitici (sinistra) VS \textit{Microkernel} (destra)}
  \label{fig:MonolithicVSmicrokernel}
\end{figure}
\newpage

\section{Scheduling della CPU}
Solitamente con il solo termine \textit{scheduling} si intende quello a breve termine della CPU, cioè la funzionalità che determina quale tra i processi (\textit{thread}) in attesa della CPU la otterranno. Chiaramente ci sono vari metodi per fare ciò, che prendono il nome di politiche di \textit{scheduling}, i quali si differenziano per modalità e prestazioni. Gli algoritmi che traducono questi metodi si chiamano algoritmi di \textit{scheduling}.

Una particolare politica di \textit{scheduling} rilevante per questo testo è \textit{Round Robin} o \textit{scheduling circolare}: consiste nel determinare un \textit{time slice} (quanto di tempo) nel quale i processi ottengono la CPU. Una volta esaurito questo tempo, il processo viene interrotto e inserito in fondo alla coda dei pronti. In questo modo tutti i processi ottengono la CPU per un tempo massimo stabilito; inoltre è possibile stabilire il tempo di attesa prima dell'esecuzione di ciascun processo in base al numero di processi che lo precedono.

\section{Memoria virtuale}
Un altro concetto fondamentale quando si parla di sistemi operativi è la gestione della memoria. Il SO deve garantire che ogni programma abbia a disposizione la giusta quantità di memoria necessaria per l'esecuzione, ed inoltre ognuno di essi deve accedere solo alla memoria a lui riservata. Un meccanismo adottato che accomuna quanto appena detto è quello di memoria virtuale.

La memoria virtuale è un meccanismo che permette di simulare uno spazio di memoria centrale (memoria primaria) maggiore di quello fisicamente presente o disponibile, dando l'illusione all'utente di un enorme quantitativo di memoria. Questa tecnica porta con sé diversi vantaggi: uno tra questi la sicurezza dovuta all'isolamento della memoria; la possibilità di condivisione di alcune pagine di memoria tra più processi (es: le pagine contenenti le librerie possono essere usate in contemporanea da più processi senza conflitti) e infine l'ultimo ma allo stesso tempo il principale vantaggio: avere a disposizione molta più memoria centrale di quella che in realtà è disponibile. 

Giustamente viene da chiedersi come tutto ciò sia possibile e il meccanismo alla base è quello di utilizzare una memoria ausiliaria, solitamente la memoria di massa, per allocare una certa parte di memoria che non è stata utilizzata recentemente. Nel momento in cui viene richiesto nuovamente la porzione di dati salvati nella memoria ausiliaria (oppure si libera spazio nella memoria centrale) i dati relativi vengono prelevati e copiati nuovamente in memoria centrale: questo processo prende il nome di \textit{swapping}. 

In presenza di memoria virtuale dunque non parleremo semplicemente di indirizzi di memoria, ma avremo una differenziazione tra indirizzi logici e indirizzi fisici. I programmi lavoreranno solo con indirizzi logici (quindi viene anche facilitata la programmazione) e poi a livello di CPU avverrà un processo di traduzione negli indirizzi fisici.

\section{Hypervisor}
Un tema che si distacca un po' dai concetti di base dei sistemi operativi ma che è rilevante per la comprensione del successivo capitolo è quello di \textit{hypervisor}. Chiamato anche \textit{virtual machine monitor} (VMM), è un tipo di \textit{sotware/firmware}, che permette di creare ed eseguire macchine virtuali. Un computer sul quale un \textit{hypervisor} esegue una o più macchine virtuali prende il nome di \textit{host machine}, mentre le singole macchina virtuali prendono il nome di \textit{guest machine}. Su ognuna è possibile eseguire un sistema operativo (anche diverso) in modo che questi siano isolati tra di loro; inoltre, al contrario di un emulatore, eseguirà la maggior parte delle istruzioni direttamente sulle risorse \textit{hardware} virtualizzate rese disponibili dall'\textit{hypervisor}.